\chapter{Conclusion}
\label{cha:conclusion}
SpiNNaker is a highly distributed hardware architecture originally designed to simulate the interaction between billions of neurons in the brain. As described in chapter \ref{cha:background}, the SpiNNaker hardware is composed of a grid of hundreds of processors with the ability to communicate over different protocols and collectively reach the solution to a problem.

This project involved making use of such hardware to build a Database Management System, namely SpiDB, divided into a NoSQL Key-Value store and an SQL Relational Database. The implementation was tested in various ways in order to find the optimal solution for processing database queries on SpiNNaker, such as the use of both a \textit{naive} and \textit{hash} storage strategies, covered in chapter \ref{cha:development}.

This project aims to explore SpiNNaker from a perspective it was not built for, which in this case is a database system. This allows benchmarking and evaluation of results which outline the architecture's strengths and weaknesses under a non-neuromorphic application. Such evaluation may be significant feedback for the current SpiNNaker research.

In the context of a general purpose application, as discussed on chapter \ref{cha:eval}, the main SpiNNaker limitations demonstrated by this project were:
\begin{itemize}
\setlength\itemsep{-0.1em}
	\item SDRAM bandwidth saturation with concurrent memory access
	\item Central point of failure and slow-down with Ethernet communication
	\item Frequent packet drops with bursts of SDP and P2P packets
	\item Limited core private memory DTCM to store incoming packets and data
\end{itemize}

The design decisions and approaches proposed by me have shown themselves to not be able to fully overcome the communications unreliability issue and the central point of failure, but insights and analysis are provided for further improvements. The evaluation presented can be used as guidelines for the development of any general purpose application on SpiNNaker and even for neuromorphic applications themselves.

SpiNNaker has also demonstrated strengths which a general purpose application can profit from:
\begin{itemize}
\setlength\itemsep{-0.1em}
	\item High scalability on number of cores and boards when arranged as a tree structure
	\item Low network latency for a large number of cores
\end{itemize}

The project aim has been successfully reached and important achievements were made on the way.

\section{Achievements}
This year long project allowed me to not only build an application which may positively impact the SpiNNaker research, by tackling it with challenges it was not designed for, but also granted me with valuable achievements, both personal and professional:

\begin{itemize}
	\item Collaboration on an IEEE article publication, alongside Alan Stokes, Thomas Heinis and Yordan Chaparov, currently under pending approval, namely \textit{Exploring the Energy Efficiency of Neuromorphic Hardware for Data Management}.
	\item Contribution to a large, international, research project on emerging hardware.
	\item Enhanced knowledge on low-level C and Python, languages which I previously had very little experience with.
\end{itemize}

\section{Future work}
This project has been the tip of the iceberg of what a fully-functional SpiNNaker database management system can be. All code written by me is now part of the official open-source SpiNNaker Software Stack. This means SpiDB is likely to expand in the future or serve as an example application running on SpiNNaker, accessible by researchers and developers around the globe. A large amount of my code is being currently used by the Imperial College London professor Thomas Heinis and the student Yordan Chaparov, who plan on improving it while decreasing power consumption. The topic of \textit{Databases on SpiNNaker} has recently had great attention within the SpiNNaker team, which may result on it becoming a full Ph.D research.

The most significant features for future research I evaluate to be:

\begin{itemize}
	\item Improving \textbf{scalability} and testing on the large scale \textit{million core machine}.
	\item Enhancing \textbf{reliability} and increasing \textbf{query sizes}, perhaps by implementing a protocol on top of the SDP and MC layer. As of now packets are limited to 256-bytes, with unreliability during busy times.
\end{itemize}

These would certainly take full advantage of the SpiNNaker hardware for hosting a Database Management System, bringing useful feedback and value.

SpiNNaker databases should also have the following features, sorted in decreasing order of importance given the project's aims:

\begin{itemize}
	\item \textbf{Caching} and \textbf{indexing}: frequently accessed areas of shared SDRAM memory can be cached at the smaller but much faster private DTCM.
	\item \textbf{Self balancing} during idle times. While no queries are being executed, cores could distribute their contents in a balanced way for faster retrieval. Indexing or other pre-processing could also be executed on the meantime.	
	\item \textbf{Additional operations} supporting table merges, triggers and aggregations.
	\item \textbf{Security} and \textbf{multi-user access}: different sections of the database can have restricted access through credentials checking.
	\item An application \textbf{server} allowing queries to be requested over the internet from different locations.
	\item A user-friendly, powerful \textbf{Graphical User Interface}
\end{itemize}