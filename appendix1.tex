\chapter{Experiments}
This chapter describes the machine specifications and code instances used to run the experiments and evaluations of this report.

\section{Specification}
\label{sec:appendix-specs}
All experiments and instances of SpiDB reported on this dissertation were run on an ASUS X555L quad core Intel(R) Core(TM) i5-5200U CPU @ 2.20GHz, 8gb DDR3 RAM @ 1600 MHz, 64-bit Ubuntu 15.04 machine connected over Ethernet to a 4-chip SpiNN-3 board, produced in September 2011.

The \textit{Memcached} and \textit{Redis} benchmarks were run under such specifications, as well as the transmission rate experiments.

\section{Communication Reliability}
\label{sec:appendix_comm_rel}

\subsection{Sending \& receiving SDP packets}

This section contains code snippets to send SDP packets from single source to single destination with variable delays/waits between them. These were used in the experiments described on section \ref{sec:eval_comm_rel}, using core 2 and 1 of chip 0,0.

C code to send SDP packets with variable delay to a single core:

\begin{lstlisting}[caption={Source}]
void send_sdp_packets(uint32_t number_of_packets, uint32_t delay, uint32_t chip, uint32_t core){
	sdp_msg_t msg;
    msg.flags       = 0x87;
    msg.tag         = 0;

    msg.srce_addr   = spin1_get_chip_id();
    msg.srce_port   = (SDP_PORT << PORT_SHIFT) | spin1_get_core_id();

    msg.dest_addr   = chip;
    msg.dest_port   = (SDP_PORT << PORT_SHIFT) | core;
    msg.length 		= sizeof(sdp_hdr_t);

    for(uint i = 0; i < number_of_packets; i++){
    	if(!spin1_send_sdp_msg(&msg, SDP_TIMEOUT)){
        	log_error("Failed to send");
        }
        sark_delay_us(delay);
    }
}
\end{lstlisting}

C code to log how many SDP packets were successfully received at destination:

\begin{lstlisting}[caption={Destination}]
uint rcv = 0;

void sdp_packet_callback(register uint mailbox, uint port) {
    rcv++;
    spin1_msg_free((sdp_msg_t*)mailbox);
    return;
}

...

spin1_callback_on(SDP_PACKET_RX,    sdp_packet_callback, 0);

\end{lstlisting}


\subsection{Storing packets}
The following code snippet (listing \ref{lst:sdp_packet_callback}) describes an efficient way of storing incoming SDP packets, to reduce packet drop, and process them under a lower priority, as discussed in section \ref{sec:eval_comm_rel}.

\lstinputlisting[language=C, caption=Storing incoming packets into a queue, label={lst:sdp_packet_callback}]{code/sdp_packet_callback.c}

\clearpage
\newpage

\subsection{Plot data}
The following table contains data from multiple runs of SpiDB instances with 100,000 \textit{put} operations under different transmission delays, used to produce figure \ref{fig:transmission-delay}.

\vspace{5mm}

\begin{tabular}{ r | r | r }
\textbf{interval ($\mu$s)} & \textbf{packet drop (\%)} & \textbf{performance (ops/sec)}  \\
100 & 0.076 	& 6175 \\
90	& 0.670		& 6608 \\
80	& 0.475		& 7042 \\
70	& 0.445		& 7408 \\
60	& 0.150		& 8394 \\
50	& 0.265		& 9057 \\
40	& 1.010		& 9882 \\
30	& 11.510	& 9918 \\
20	& 22.910	& 9506 \\ 
10	& 34.140	& 8960 \\
7	& 33.965 	& 9778 \\
4	& 98.432 	& 243 \\
\end{tabular}

\section{Query formats}
\label{sec:appendix-queries}
This section contains the query structs and enums used for internal and external communication of the database. The two supported variable types are: \textit{int} and \textit{char[]} (string).

\large Types:

\lstinputlisting[language=C]{code/types.c}

\large Key-value structs:

\lstinputlisting[language=C]{code/db_type_kv.c}

\large SQL database structs:

\lstinputlisting[language=C]{code/db_type_relational.c}