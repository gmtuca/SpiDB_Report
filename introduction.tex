\chapter{Introduction}
\label{cha:intro}
This chapter describes a high level view of the SpiNNaker platform and its main uses. It highlights also the motivation and aims of my project and how it may impact on improvement to a large scale international research.

\section{SpiNNaker Overview}
\label{sec:overview}
"SpiNNaker (Spiking Neural Network Architecture) is a biologically inspired, massively parallel computing engine designed to facilitate the modelling and simulation of large-scale spiking neural networks of up to a billion neurons and trillion synapses (inter-neuron connections) in biological real time." \cite{painkras} The SpiNNaker project, inspired by the fundamental structure and function of the human brain, began in 2005 and it is a collaboration between several universities and industrial partners: University of Manchester, University of Southampton, University of Cambridge, University of Sheffield, ARM Ltd, Silistix Ltd, Thales \cite{spinnproject}. A single SpiNNaker board is composed of hundreds of processing cores, allowing it to efficiently compute the interaction between populations of neurons.

\section{Project Aim}
\label{sec:aim}
This research project involves making use of the SpiNNaker software stack and hardware infrastructure, both optimised for neural network simulations, in order to explore and evaluate its usability and performance as a general purpose platform. This has been achieved through the development of a distributed Key-Value store and a simple Relational Database Management System (DBMS). This open-source experimental project has grown to be the \textit{largest non-neuromorphic application in the SpiNNaker API} (SpiNNaker Software Stack), available widely for further research.

SpiNNaker was designed from the outset to support parallel execution of application code while ensuring energy efficiency \cite{discrete}. This appealed as an extraordinary opportunity to store and retrieve data in a fast, distributed way, under a DBMS. This allows exploration of a broad range of ideas outside of the initial scope of SpiNNaker, testing some of its capabilities and limitations against a non-neuromorphic application, bringing useful feedback to the team.

Given such opportunities, the main aims of this project are:

\begin{itemize}
	\item \textbf{Testing usability} of SpiNNaker as a general purpose platform, exploring its strengths and weaknesses under an environment it was not originally designed for.
	\item Performing \textbf{application benchmarks}, allowing analysis which can provide insights for design decisions to the current architecture and possibly influence on the structure of the next generations of SpiNNaker chips.
\end{itemize}